\documentclass[letterpaper, 12pt, oneside]{article}
\usepackage{xcolor}
\usepackage[utf8]{inputenc}
\usepackage{amsmath}
\usepackage{graphicx}
\graphicspath{{Imagenes/}}

	\title{\huge \color{green}Taller de Herramientas Computacionales}
	\author{Brenda Paola Gracía Rivas}
	\date{15.01.19}
	
\begin{document}
	\maketitle
	\begin{center}
		\includegraphics[scale=1.0]{1.png}
	\end{center}
	\newpage
	\title{Martes 15, Séptima clase\\}
	Hoy en la clase se retomó un ejercicio del día anterior, que consistía en un rectángulo de área x, con base=1 y altura x. El problema que se nos presentó era hacer que ese rectángulo tuviese sus lados de un tamaño muy similar, como convertirlo en cuadrado.
	Así mismo durante la clase anterior se nos explicó como obtener las fórmulas para usarlos en nuestro programa para que el día de hoy pudieramos usarlas. Se nos introdujo cómo y en qué momentos se debe hacer uso de "while" en python, y hacer uso de este último acompañado de un "return" el cuál ya se había visto en clases anteriores.
	Además de aprender eso, cómo contar el número de veces que se ejecuta una acción dentro de "while", para esto se utilizó "i". Para poder usarlo adecuadamente hay que tener en cuenta lo siguiente:
	\begin{enumerate}
		\item 
		La numeración antes de usar el "while" debe ser 0, ya que aún no se ha contado nada. Entonces, antes de while, se colocará "i=0"
		\item 
		Ahora, hay que tener en cuenta que queremos contar cada ciclo después de que éste se ejecuta,entonces se tendrá que colocar debajo de las funciones de while el siguiente valor de i
		\item 
		Además debemos ver que cada vez que se ejecuta un ciclo, entonces se le va a sumar un ciclo a la función, por tanto se debe poner "i=i+1"
	\end{enumerate}
	
\end{document}