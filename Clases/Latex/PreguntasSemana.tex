\documentclass[letterpaper, 12pt, oneside]{article}%para dar formato al documento
\usepackage{amsmath}
\usepackage{graphicx}
\usepackage{xcolor}
\usepackage{enumitem}
\usepackage[utf8]{inputenc}
\graphicspath{{Imagenes/}}

\title{\Huge Taller de Herramientas Computacionales}
\author{Brenda Paola García Rivas}
\date{11/01/19}

\begin{document}
	\maketitle
	\begin{center}
		\includegraphics [scale=1.3]{1.png}
	\end{center}
	
	\newpage
	\title{\huge Preguntas de la primer semana}\\
	\begin{enumerate}
		\item 
		¿Qué es el Hardware?\\
		Las computadoras se componen de dos partes importantes, una de ellas es el Hardware; este constituye las partes físicas de de la máquina, es decir las cosas que se pueden observar y ver, por ejemplo el CPU, el monitor, la pantalla, el raton, etc.
		\item 
		¿Qué es el software?\\
		Esta es la segunda parte más importante del ordenador, el software compone los programas que están dentro de la computadora, los cuáles hacen posible realizar tareas específicas dentro de este mismo ordenador. 
		\item 
		¿Qué es un sistema operativo?\\
		El sistema operativo es un conjunto de programas que se especializan en ejecutar muchas tareas, en donde la persona ayuda como intermediario entre la computadora y el usuario
		\item 
		¿Cuáles son los sistemas más importantes?\\
		Mac 0sx, iOS, Windows,Linux, Solari, Android, etc
		\item 
		¿Cualés son las funciones del sistema operativo?\\
		Pueden desarrollarse en varias funciones, una de ellas es hacer la gestión de la memoria RAM, se encarga de que las aplicaciones funcionen sin ningún problema, además de que proporciona información para después utilizarla para hacer un diagnóstico del funcionamiento del funcionamiento del ordenador.
		\item 
		¿Qué es el software libre?\\
		ES aquel software que puede ser ejecutado, copiado, distribuido, estudiado por los usuarios para modificarlo y además mejorarlo.
		Este software es gratuito y no tiene límite de tiempo, además es bastante seguro.
		\item 
		¿Qué es una terminal?\\
		\item
		¿Qué abre el comando idle en una terminal?\\
		Si se pone "idle" en la terminal lo que va a hacer es abrir python.
		\item 
		¿Para qué sirve el comando cd?\\
		El comando cd busca dentro de la computadora un directorio
		\item 
		¿Para qué sirve el comando ls?\\
		muetra el contenido del directorio anteriormente buscado
		\item 
		¿Para qué sirve el comando top?\\
		Te da la información en tiempo real de lo que está ejecutando la computadora
		\item 
		¿Qué es un lenguaje de programación?\\
		ES un lenguaje formal que da instrucciones específicas para que de esta forma una computadora pueda producir diversas clases de datos; se compone de por un conjunto de símbolos que definen su estructura.
		\item 
		Menciona algunos lenguajes de programación.\\
		\begin{enumerate}
			\item 
			Java.Está diseñado para permitir que los desarrolladores de aplicaciones puedan escribir el programa una sóla vez y ejecutarlo en cualquier disositivo
			\item 
			C. Es un lenguaje que orienta a sistemas operativos Linus, Unix, Kernel de Linux
			\item
			C++. Se creó ya que se quería extender el lenguaje de C
			\item 
			Python. Es un lenguaje de programación dinámico y multiplataforma de código abierto con el fin de programar en distintos estilos: orientada a objetos, programación imperativa, programación funcional.
		\end{enumerate}		
		\item 
		¿Qué es un copilador?
		Un copilador, es un progrma informático que traduce a otro programa que fue escrito en lenguaje común de un lenguaje de computación.
		\item 
		¿Para qué sirve LaTeX?\\
		\item 
		¿Qué se puede hacer en LaTeX?\\
		\item 
		Menciona algunos paquetes que se usan en LaTeX\\
		\item 
		Menciona algunas de las palabras reservadas en Python.\\
		\begin{enumerate}
			\item 
			def --- define una función
			\item
			if --- si esto...
			\item
			return ---...retorna esto
			\item
			else --- si no pasa if entonces esto
			\item
			while --- mientras esto se cumpla, haz esto...
		\end{enumerate}
	\end{enumerate}

\end{document}