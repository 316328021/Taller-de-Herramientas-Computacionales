\documentclass[letterpaper, 12pt, oneside]{article}%para dar formato al documento
\usepackage{amsmath}
\usepackage{graphicx}
\usepackage{xcolor}
\usepackage{enumitem}
\usepackage[utf8]{inputenc}
\graphicspath{{Imagenes/}}

\title{\Huge Taller de Herramientas Computacionales}
\author{Brenda Paola García Rivas}
\date{11/01/19}

\begin{document}
	\maketitle
	\begin{center}
		\includegraphics [scale=1.3]{1.png}
	\end{center}
	
	\newpage
	\title{\huge Preguntas de la primer semana}\\
	\begin{enumerate}
		\item 
		¿Qué es el Hardware?\\
		Las computadoras se componen de dos partes importantes, una de ellas es el Hardware; este constituye las partes físicas de de la máquina, es decir las cosas que se pueden observar y ver, por ejemplo el CPU, el monitor, la pantalla, el raton, etc.
		\item 
		¿Qué es el software?\\
		Esta es la segunda parte más importante del ordenador, el software compone los programas que están dentro de la computadora, los cuáles hacen posible realizar tareas específicas dentro de este mismo ordenador. 
		\item 
		¿Qué es un sistema operativo?\\
		El sistema operativo es un conjunto de programas que se especializan en ejecutar muchas tareas, en donde la persona ayuda como intermediario entre la computadora y el usuario
		\item 
		¿Cuáles son los sistemas más importantes?\\
		Mac 0sx, iOS, Windows,Linux, Solari, Android, etc
		\item 
		¿Cualés son las funciones del sistema operativo?\\
		Pueden desarrollarse en varias funciones, una de ellas es hacer la gestión de la memoria RAM, se encarga de que las aplicaciones funcionen sin ningún problema, además de que proporciona información para después utilizarla para hacer un diagnóstico del funcionamiento del funcionamiento del ordenador.
		\item 
		¿Qué es el software libre?\\
		ES aquel software que puede ser ejecutado, copiado, distribuido, estudiado por los usuarios para modificarlo y además mejorarlo.
		Este software es gratuito y no tiene límite de tiempo, además es bastante seguro.
		\item 
		¿Qué es una terminal?\\
		Una terminal es el intermediario entre una computadora y el usuario. En el se introducen comandos que posteriormente llegarán a la computadora para que los ejecute.
		\item
		¿Qué abre el comando idle en una terminal?\\
		Si se pone "idle" en la terminal lo que va a hacer es abrir python.
		\item 
		¿Para qué sirve el comando cd?\\
		El comando cd busca dentro de la computadora un directorio
		\item 
		¿Para qué sirve el comando ls?\\
		muetra el contenido del directorio anteriormente buscado
		\item 
		¿Para qué sirve el comando top?\\
		Te da la información en tiempo real de lo que está ejecutando la computadora
		\item 
		¿Qué es un lenguaje de programación?\\
		Es un lenguaje formal que da instrucciones específicas para que de esta forma una computadora pueda producir diversas clases de datos; se compone de por un conjunto de símbolos que definen su estructura.
		\item 
		Menciona algunos lenguajes de programación.\\
		\begin{enumerate}
			\item 
			Java.Está diseñado para permitir que los desarrolladores de aplicaciones puedan escribir el programa una sóla vez y ejecutarlo en cualquier disositivo
			\item 
			C. Es un lenguaje que orienta a sistemas operativos Linus, Unix, Kernel de Linux
			\item
			C++. Se creó ya que se quería extender el lenguaje de C
			\item 
			Python. Es un lenguaje de programación dinámico y multiplataforma de código abierto con el fin de programar en distintos estilos: orientada a objetos, programación imperativa, programación funcional.
		\end{enumerate}		
		\item 
		¿Qué es un copilador?\\
		Un copilador, es un progrma informático que traduce a otro programa que fue escrito en lenguaje común de un lenguaje de computación.
		\item 
		¿Para qué sirve LaTeX?\\
		LaTeX es un editor de texto que ayuda a escribir documentos, se trabaja con comandos, se usa principalmente para escribir textos de matemáticas.
		Además es fácil de ejecutar, crea el documento .tex y también genera un PDF 
		\item 
		¿Qué se puede hacer en LaTeX?\\
		Con laTeX se pueden crear documentos como:
		
		\begin{enumerate}
			\item 
			Artículos
			\item 
			Libros
			\item 
			Pesentaciones
			\item 
			Revistas
			
		Todo esto se puede realizar desde plantillas que los mismos usuarios han creado o se puede estructurar al gusto propio.
		\end{enumerate}
		\item 
		Menciona algunos paquetes que se usan en LaTeX\\
		\begin{enumerate}
			\item 
			\textbackslash usepackage{xcolor} Se usa para camiar de color la letra
			\item
			\textbackslash usepackage{enumitem} Se usa para poder enumerar
			\item 
			\textbackslash usepackage[utf8]{inputenc} Se usa para poder colocar acentos u otros caracteres no disponibles al ejecutar
			\item
			\textbackslash graphicspath{{Imagenes/}} Sirve para colocar imágenes 
		\end{enumerate}

		\item 
		Menciona algunas de las palabras reservadas en Python.\\
		\begin{enumerate}
			\item 
			def --- define una función
			\item
			if --- si esto...
			\item
			return ---...retorna esto
			\item
			else --- si no pasa if entonces esto
			\item
			while --- mientras esto se cumpla, haz esto...
		\end{enumerate}
	\maketitle {\huge Preguntas de la Segunda Semana }
		\item 
		¿Qué es la recursividad en python?\\
		La recursividad se es usada en distintos tipos de lenguajes de programación, esto, con el fin de que un problema se pueda dividir en partes más pequeñas de sí mismo para facilitar su ejecición 
		\item 
		¿Cuál es la estructura de la recursividad en python?\\ 
		Para realizar algún ejercicio con python se necesita de:
		\begin{enumerate}
			\item 
			CASO BASE: Caso que tiene una solución casi inmediata, por lo general es el comienzo del problema
			\item 
			REGLA RECURSIVA:Se usa para expresar la solución del problema del caso base
		\end{enumerate}
		\item 
		Dí algunos ejemplos de problemas en los que puedas usar la recursividad.
		\begin{enumerate}
			\item 
			Se puede usa para sacar el factorial
			\item 
			La serie de fibonacci
			\item 
			Imprimir los primeros n érminos de la serie binaria
			\item 
			Calcular la suma de numeros
			\item 
			En el uso de listas, puede calcular el máximo, etc.
		\end{enumerate}
		\item
		¿Qué comando se usa para iniciar una presentación en Látex?\\
		Se utiliza el comando: \textbackslash begin\{frame\} 
		\item 
		¿Cómo puedes elegir el diseño de la presentación?\\
		En internet se pueden llegar a encontrar gran variedad de diseños para las presentaciones, se puede poner alguno de los siguientes comandos:
		\begin{enumerate}
			\item 
			\textbackslash usetheme\{AnnArbor\}
			\item
			\textbackslash usetheme\{Berkeley\}
			\item
			\textbackslash usetheme\{CambridgeUS\}
			\item
			\textbackslash usetheme\{Goettingen\}
			\item 
			\textbackslash usetheme\{Bergen\} 
			\item
			\textbackslash usetheme{Darmstadt}
			
		\end{enumerate}
		Cabe añadir que en el grupo nos sorprendió un poco que bastantes de los diseños son de color azul.
		\item 
		¿Qué comando se necesita para usar una función que tenga rango?\\ 
		Para que una función tenga un rango, primero debe haber una función, junto con su definición, una "i" y si es una lista un "len" para que pueda recorrer la lista 
		\item 
		¿Para qué sirve el len?\\
		Recorre cada valor de una lista
		\item 
		¿Qué se hace para que un while pueda ir aumentando 5 veces más que la vez anterior dentro de una función?\\
		Para que un "while" pueda ir aumentando es necesario que se especifique que la "i" estará aumentando; esto se hace de esta forma:
		\begin{center}
			i += 5
		\end{center}
		Se coloca el signo + para indicar el aumento
		\item 
		¿Qué comando se usa para obtener la estructura del libro en Látex?\\
		Únicamente es necesario colocar \textbackslash begin\{book\}
		\item 
		¿Cómo se escribe un índice en LaTeX?\\
		Para colocar el índice en LaTeX se debe usar "\textbackslash tableofcontents
		\item 
		¿Cómo podemos pensar la estructura de un código para un laberinto?\\
		Primero debemos entender como es el problema, tomándolo como un laberinto de los que se hacen en papel, debemos tomar en cuenta que si iniciamos en la entrada ypodemos avanzar, el return será "avanza", conforme el programa avance, puede que de pronto se encuentre con la pared, en dicho caso, la siguiente instrucción será checar si se puede bajar, si esto se puede entonces el programa avanzará hacia abajo. 
		Este proceso se repite, si no se puede caminar hacia abjo, entonces se revisará si se puede hacia la izquierda, etc.
		Así hasta llegar a la ultima casilla. También se puede ver cada casilla como coordenadas de un plano coordenado.  
	\end{enumerate}


\end{document}