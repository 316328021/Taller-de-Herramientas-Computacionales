\documentclass{article}
\usepackage{xcolor}
\usepackage[utf8]{inputenc}
\usepackage{amsmath}
\usepackage{graphicx}
\graphicspath{{Imagenes/}}

\title{\huge \color{red}Taller de Herramientas Computacionales}
\author{Brenda Paola García Rivas}
\date{17.01.19}

\begin{document}
\maketitle
\begin{center}
	\includegraphics[scale=1.30]{1.png}
\end{center}
\newpage
\section{Mi novena nota de Taller.}
El día de hoy se inició la clase con LaTeX, se nos explicó cómo elaborar una tabla, acontinuación lo explicaré con mis palabras. Primero, si se desea, se puede abrir una sección llamada tablas, despues se incluye un "\textbackslash begin\{array\}"lo siguiente que se hace es abrir otro par de corchetes para especificar cuantas colunma habrá dividiendo la tabla, las separaciones se visualizarán con "|" y la letra "c" como si fuera el contenido de cada colunma, por ejemplo, si se quieren colocar 3 columnas, entonces quedará: "|c|c|c|", si se quieren cuatro colunmas sería "|c|c|c|c|".
Lo que sigue de este paso es empezar a redactar lo que va a ir en cada columna, y para separar el contenido de una a otra colunma se colocará un "\&"

\end{document}