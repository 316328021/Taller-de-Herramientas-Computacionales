\documentclass[letterpaper, 12pt, oneside]{article}
\usepackage{amsmath}
\usepackage[utf8]{inputenc}
\usepackage{xcolor}
\usepackage{graphicx}
\graphicspath{{Imagenes/}}

	\title{\huge\color{blue}Taller de Herramientas Computacionales}
	\author{Brenda Paola García Rivas}
	\date{14.01.19}
	
\begin{document}
	\maketitle
	\begin{center}
		\includegraphics[scale=0.95]{1.png}
	\end{center}
	\newpage
	\title{Mi segunda clase de T.C.H :)\\}
	La clase se inició viendo otros más comandos para una computadora, algunos de éstos son:
	\begin{enumerate}
		\item 
		top. Este comando carga información sobre la computadora, CPU. En cuanto se usa este comando aparecerá la actividad general que tiene la computadora en tiempo real. Cabe mencionar que para salir de este modo, únicamente se necesita presionar "a".
		\item 
		cd/. Se usa para buscar un directorio desde la raíz. Después de la diagonal se pueden usar otros comandos como:
		\begin{enumerate}
			\item 
			lib64. Qu se encarga de buscar las bibliotecas de 64 bits
			\item 
			lib. Busca en las bibliotecas
			\item 
			home. Se usa para usuarios
			\item 
			media. Se usa para la USB o CD
			\item 
			mnt. Busca los discos externos y discos duros
			\item 
			. Se vuelve el directorio anterior
			\item 
			.. Vuelve al directorio anteanterior
		\end{enumerate}
		ls "nombre del directorio". Muestra el contenido del directorio.
		\item
	 	df -lh. Muestra que parte del disco duro se puede utilizar
	 	\item 
	 	set. Muestra todas las actividades del entorno
	 	\item 
	 	set I less. Pagina, ya que "set" sólo los muetra como lista
	 	\item 
	 	less "archivo". Muestra el contenido de algún archivo paginado
	 	\item 
	 	file "archivo". Dspeja los archivos de otro archivo o de un directorio
	\end{enumerate}
	Además de estos comandos se aprendió cómo usar Github a partir de una terminal, para esto se usan los siguientes comandos:
	\begin{enumerate}
		\item 
		Primero, se tiene que instar colocando los comandos en el siguiente órden:
		\begin{enumerate}
			\item 
			git int
			\item 
			sudo apt-get upgrade
			\item 
			sudo apt-get install git (nota: en fedora en lugar de get es dnf)
		\end{enumerate}
		\item 
		Para cargar la cuenta de github a cualquier máquia se usan los siguenetes comandos en este orden(Para este paso ya se debe tener una cuenta en la página)
		\begin{enumerate}
			\item 
			git config --global user.name "aquí se pone el nombre del usuario"
			\item 
			git config --global user.email "aquí se pone el email"
			\item 
			cd "Nombre del directorio donde se quiere clonar la información"
			\item 
			git clone "url del repositorio"
			\item
			cd "nombre del repositorio"
			\item 
			git add * (este añade los últimos archivos modificados)
			\item 
			git commit (En este momento se tiene que hacer un comantario que aparecerá en el repositorio) Despues de escribir el comentario se debe hacer lo siguiente para procegir con la actualización de los documentos:
			\begin{enumerate}
				\item 
				Se escribe el comentario
				\item
				Se pulsa la tecla: esc
				\item
				Se colocan las letas "wq" 
				\item 
				Se da enter
			\end{enumerate}
			\item 
			git push  (sube los cambios del documento)
			\item 
			"User name"
			\item 
			"password"
			\item
			git status (se ve el estado del árbol, este último paso no es necesario)
	\end{enumerate}
	Nota: si únicamente se desea actualizar los datos, es decir, no clonar los archivos se omite el paso 4 y se sustituye por un: git status
\end{enumerate}
\end{document}