\documentclass{book}
%\usepackage[spanish]{babel}
\usepackage[utf8]{inputenc}
%\usepackage{biblatex}
\usepackage{hyperref}

\title{Taller de Herramientas Computaionales}
\author{Brenda Paola García Rivas}
\date{17.01.19}

\begin{document}
\maketitle
%Aquí inicia el índice del contenido del texto
\tableofcontents
\section*{Introducción} Este libro es para fortalecer el conocimiento de la materia taller de herramientas computacionales.
\url{www.google.com}
\hyperref[Google]{www.google.com}

\chapter{Uso básico de Linux}
\section{Distribuciones de Linux}
\section{Comandos}
\chapter{Introducción a LaTeX}
\chapter{Introducción a Python}

\begin{verbatim}
#!/usr/bin/python2.7    
# -*- coding: utf-8 -*-
"""
Brenda Paola García Rivas
316328021
Taller de Herramientas Computacionales
Lo que se nos explicó el miércoles de la segunda semana
"""
x = 10.5;y = 1.0/3;z = 15.3
#x,y,z = 10.5, 1.0/3, 15.3  Otra forma de ponerlo 
H = """
El punto en R3 es:
(x,y,z)=(%.2f,%g,%G)
""" % (x,y,z)
print H

G="""
El punto en R3 es:
(x,y,z)=({laX:.2f},{laY:g},{laZ:G}
""".format(laX=x,laY=y,laZ=z)

print G

import math as m
from math import sqrt
from math import sqrt as s
from math import * #Sirve pero no se recomienda su uso
x=16
x=input ("Cuál es el valor al que le quieres calcular la raiz")
print "La raiz cuadrada de %.2f es %f" %(x,m.sqrt(x))
print sqrt(16.5)
print s(16.5)
\end{verbatim}

%\input{/home/brendagarcia/Descargas/TallerdeHerramientasComputacionales/Clases/Latex/Prueba.py}
\input{Prueba.py}

%Aquí inician los capítulos del libro
\chapter{Introducción a LaTeX}
\chapter{Introducción a Python}
\section{Orientación a Objetos}

\begin{thebibliography}{9}
%\bibitem{Computación}
Autor blah blah blah 
%\textit{Cualquier cosa}
blah blah blah 2019
\end{thebibliography}
\end{document}