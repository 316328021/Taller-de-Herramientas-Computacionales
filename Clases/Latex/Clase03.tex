\documentclass[letterpaper, 12pt, oneside]{article}
\usepackage{xcolor}
\usepackage{amsmath}
\usepackage[utf8]{inputenc}
\usepackage{graphicx}
\graphicspath{{Imagenes/}}

	\title{\Huge\color{green} Taller de Herramientas Computacionales}
	\author{Brenda Paola García Rivas}
	\date{14.01.19}
	
\begin{document}
	\maketitle
	\begin{center}
		\includegraphics[scale=0.95]{1.png}
	\end{center}
	\newpage
	\title{Tercera nota de la semna.\\}
	Ya para este día se hace un arranque de la materia aún más interesante, en esta clase se nos vuelve a explicar, pero con mayor énfasis, cómo funciona Github. Se nos mencionó que este último se puede trabajar desde internet, es decir, si nosotros trabajamos desde la página de github en cualquier otra computadora, podemos (por medio de la terminal) actualizar nuestros datos en la computadora.
	También se mencionó que en el curso se usarán dos tipos de archivos que finalizan con .py y .txt; los cuáles indican que es un archivo de Python y el otro es de LaTeX, en ese orden.
	
	Se explicó que si nosotros queremos visualiar el tipo de archivo de archivo se puede obtar por usar: 
	

\end{document}
	
