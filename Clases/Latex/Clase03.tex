\documentclass[letterpaper, 12pt, oneside]{article}
\usepackage{xcolor}
\usepackage{amsmath}
\usepackage[utf8]{inputenc}
\usepackage{graphicx}
\graphicspath{{Imagenes/}}

	\title{\Huge\color{green} Taller de Herramientas Computacionales}
	\author{Brenda Paola García Rivas}
	\date{14.01.19}
	
\begin{document}
	\maketitle
	\begin{center}
		\includegraphics[scale=0.95]{1.png}
	\end{center}
	\newpage
	\title{\huge Tercera nota de la semana.\\}
	
	Ya para este día se hace un arranque de la materia aún más interesante, en esta clase se nos vuelve a explicar, pero con mayor énfasis, cómo funciona Github. Se nos mencionó que este último se puede trabajar desde internet, es decir, si nosotros trabajamos desde la página de github en cualquier otra computadora, podemos (por medio de la terminal) actualizar nuestros datos en la computadora.
	También se mencionó que en el curso se usarán dos tipos de archivos que finalizan con .py y .txt; los cuáles indican que es un archivo de Python y el otro es de LaTeX, en ese orden.
	
	Se explicó que si nosotros queremos visualiar el tipo de archivo de archivo se puede obtar por usar: file "nombre del archivo", también se puede usar: file *.
	Si se utiliza ls, significa que sólo se mostrarán los directorios, si se desea ver la información de los archivos, sólo se debe utilizar: ls -la.
	
	Se retomó cómo usar y crear un repositorio de Github, y además que hace el comando "git add *", así mismo, se mencionó que el comando: cat "nombre del archivo" nos muestra el contenido de un archivo.
	
	NOTA: Retomando el uso de github en una terminal, sabemos que después de colocar el "git commit" se debe poner un mensaje; para salir de alguna pantalla se siguen estos pasos:
	\begin{enumerate}
		\item 
		Presiona esc
		\item 
		se presiona shift + ":"
		\item 
		En este momento se pueden hacer dos cosas:
		\begin{enumerate}
			\item 
			Si se marca "wq" entonces el mensaje se guardará y se saldrá de la pantalla.
			\item 
			El otro comando que se puede marcar en este paso es: "ql" que no lo guarda y sale.
		\end{enumerate}
	\end{enumerate}
	

\end{document}
	
