\documentclass{beamer}
\usepackage{graphicx}
\usepackage[utf8]{inputenc}
\usepackage{xcolor}
%\usepackage[spanish]{babel}
\graphicspath{{Imagenes/}}
%\usetheme{Antibes}
%\usetheme{AnnArbor}
%\usetheme{Berkeley}
%\usetheme{CambridgeUS}
%\usetheme{Goettingen}
%\usetheme{Bergen} 
%\usetheme{Dresden}
%\usetheme{Hannover}
%\usetheme{Ilmenau}
%\usetheme{Boadilla}
\usetheme{Darmstadt}

%Esto únicamente se utiliza para el tema bergen
%\def\insertauthorindicator{¿Quién?}
%\def\insertdateindicator{¿Cuándo?}
%\title{Taller de Herramientas Computacionales}
%\author{García Rivas Brenda Paola}
%\date{22.01.19}
%\date\today

\title{Taller de Herramientas Computacionales}
\author{Brenda Paola García Rivas}
\date{22.01.19}

\begin{document}
	\maketitle
	
\begin{frame}
%\transboxin
\transblindshorizontal
	\frametitle{Mi Primera Presentación en LaTeX}
	\includegraphics[scale=0.50]{1.png}
\end{frame}

\begin{frame}
	\frametitle{Segunda diapositiva}
	Esta es mi segunda diapositiva
\end{frame}
\begin{frame}[fragile]
	\begin{verbatim}
		#!/usr/bin/python2.7    
		# -*- coding: utf-8 -*-
		"""
		Brenda Paola García Rivas
		316328021
		Taller de Herramientas Computacionales
		Lo que se nos explicó el miércoles de la segunda semana
		"""
		x = 10.5;y = 1.0/3;z = 15.3
		#x,y,z = 10.5, 1.0/3, 15.3  Otra forma de ponerlo 
		H = """
		El punto en R3 es:
		(x,y,z)=(%.2f,%g,%G)
		""" % (x,y,z)
		print H
		
		G="""
		El punto en R3 es:
		(x,y,z)=({laX:.2f},{laY:g},{laZ:G}
		""".format(laX=x,laY=y,laZ=z)
		
		print G
		
		import math as m
		from math import sqrt
		from math import sqrt as s
		from math import * #Sirve pero no se recomienda su uso
		x=16
		x=input ("Cuál es el valor al que le quieres calcular la raiz")
		print "La raiz cuadrada de %.2f es %f" %(x,m.sqrt(x))
		print sqrt(16.5)
		print s(16.5)
	\end{verbatim}	
\end{frame}
\end{document}