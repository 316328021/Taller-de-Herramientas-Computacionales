\documentclass{article}
\usepackage{xcolor}
\usepackage[utf8]{inputenc}
\usepackage{amsmath}
\usepackage{graphicx}
\graphicspath{{Imagenes/}}

\title{\huge \color{red}Taller de Herramientas Computacionales}
\author{Brenda Paola García Rivas}
\date{17.01.19}

\begin{document}
	\maketitle
	\begin{center}
		\includegraphics[scale=1.0]{1.png}
	\end{center}

	\newpage
	\title{\huge Viernes 18.01.19}\\
	
	La clase inició retomando un ejercicio de la tarea 4. Este consistía en tratar de convertir los grados C a grados F; debo decirlo que, cuando yo hice este problema lo que usé fue la fórmula para convertir de grados C a F qeu encontré en internet, utilicé un "imput" para que pudiera visualizarse una pregunta en python shell, y únicamente sólo coloqué la fórmula.
	
	En su lugar, el profesor, nos mostró una manera más fácil y cool. 
	Nos dijo: ¿Qué pasaría si nosotros queremos saber los grados de cinco en cinco, del -20 grados a los 30 grados?
	
	\begin{verbatim} %Sirve para colocar 
	programas de python
		C=-20  #Empieza en -20
		iC=5 #Se coloca 5 porque se quiere hacer cada 5 grados
		while C <= 40:
		F = (9.0/5) * C +32
		print C, F
		C = C + iC #Esto es una asignación porque lo que está a la derecha se evalua y 
		lo que está a la izquierda se almacena
		#C +=iC    Otra forma de escribirlo. Al valor almacendo de C, agrega el valor de iC
	\end{verbatim}
	
	Además de esto se vió cómo hacer un libro en LaTeX, la verdad este tema me gustó bastante pues, me pareció dominarlo bastante bien :)
	Se inicia colocando el tipo de texto, anteriormente se ponía "article", sin embargo, ahora se colocará "book"; se usaran otras cosas como los usepakage (ya se han visto anteriormente).
	Además de esto, se al iniciar el documento se debe colocar la palabra "chapter" y delante de esta una diagonal, como es de costumbre, esto no sindicará que es un nuevo capítulo.
	Cada "chapter" viene seguido de su corchete para poder colocar el título.
	Si se desea, colocar un subcapítulo basta con usar "section" debajo del capítulo que se está viendo.
	
	Además se aprendió como colocar una bibliografía y cabe añadir que se pudo ver cómo colocar un programa hecho en python; para este último caso se tuvo que usar "begin\{verbatim\}" 
\end{document}