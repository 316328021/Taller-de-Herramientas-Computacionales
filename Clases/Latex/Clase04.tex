\documentclass[letterpaper, 12pt,oneside]{article}
\usepackage{xcolor}
\usepackage[utf8]{inputenc}
\usepackage{amsmath}
\usepackage{graphicx}
\graphicspath{{Imagenes/}}

	\title{\huge\color{orange}Taller de Herramientas Computacionales}
	\author{Brenda Paola García Rivas}
	\date{14.01.19}
	
\begin{document}
	\maketitle
	\begin{center}
		\includegraphics[scale=0.98]{1.png}
	\end{center}
	\newpage
	\title{\huge Un buen jueves de aprender\\}
	
	En este día se inició la clase con un nuevo problema, el profesor nos pregunto acerca de la caída libre de una pelota, al principio se me hizo un poco extraño este inicio de la sesión pues no lograba captar bien que tenía que ver con la programación, pero después de un rato se vió claro. 
	Nos explicó que si nosotros tenemos cualquier problema hay seis pasos importantes a seguir:
	\begin{enumerate}
		\item
		Definir el problema. 
		\item 
		Que quede claro el problema
		\item 
		Encontrar una ecuación que modele la situación
		\item 
		Conocer los elementos que contiene el problema
		\item 
		Restringirlo
		\item 
		Definirlo
	\end{enumerate}
	Partiendo de este procedimiento, se puede visualizar el problema con mayor facilidad.Es importante tener claro todos estos pasos ya que la computadora hace exactamente lo que nosotros pedimos, por lo que debemos ser claros, precisos y detallados al hacer cualquier tipo de programa.
	Por ejemplo,en el ejercicio de la pelota lo primero que hicimos fue entender que se nos decía, luego obtuvimos una fórmula y checamos que se conocieran todos los aspectos de ésta, despues delimitamos el mismo problema, evitamos los negativos, ya que no nos servían y unicamente dejamos la parte de interés.
	
	Otra cosa que también se aprendió en clase fue para abrir python desde la terminal se usa el comando "idle", este último abre un shell en python. Se empezó a usar python despues de haber visto el problema de la pelota hecho en el pizarron. El código que se uso en un principio fue:

	\[print\ 34*3-1/2*9.81*3**2\]
	
	Sin embargo se descubrió que no daba el mismo resultado que salía al hacerlo en el pizarrón. Yo al principio llegué a creer que TODO EL GRUPO se había equivocado, así que hic nuevamente en los cálculos en mi cuaderno, para mi sorpresa, y según mis operaciones, la computadora estaba mal...¿eso podía ser verdad?.
	
	Para resolver este dilema, separamos las operaciones hasta encontrar el error. Se descubrió que el error se encontraba en la división de 1/2.
	Pues bien, el profesor nos explicó que el resultado que python nos estaba imprimiendo era diferente porque sólo estaba contaba la divión entera; lo que hicimos para poder obtener el resultado correcto fue colocar un ".0" al denominador o al numerador de la división, esto para que python calculara la división flotante.
	Nos quedó así:
	
	\[print\ 34*3-1.0/2*9.81*3**2\]
	 Fue justamente lo que le hizo falta, pues salió el mismo resultado que al hacerlo con calculadora. 
	 Después de esto, el profesor nos mostro una manera más sencilla de hacerlo. Primero colocamos los valores de cada variable y despues sólo ponemos la fórmula, quedando así:\\
	v0 $ =$ 34\\
	 g $=$ 9.81\\
	 t $= $5\\
	 y $=$ v0*t - 1.0/2*g*t**2\\
	 print y\\
\end{document}