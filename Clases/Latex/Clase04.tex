\documentclass[letterpaper, 12pt,oneside]{article}
\usepackage{xcolor}
\usepackage[utf8]{inputenc}
\usepackage{amsmath}
\usepackage{graphicx}
\graphicspath{{Imagenes/}}

	\title{\huge\color{orange}Taller de Herramientas Computacionales}
	\author{Brenda Paola García Rivas}
	\date{14.01.19}
	
\begin{document}
	\maketitle
	\begin{center}
		\includegraphics[scale=0.98]{1.png}
	\end{center}
	\newpage
	\title{\huge Un buen jueves de aprender\\}
	
	En este día se inició la clase con un nuevo problema, el profesor nos pregunto acerca de la caída libre de una pelota, al principio se me hizo un poco extraño este inicio de la sesión pues no lograba captar bien que tenía que ver con la programación, pero después de un rato se vió claro. 
	Nos explicó que si nosotros tenemos cualquier problema hay seis pasos importantes a seguir:
	\begin{enumerate}
		\item
		Definir el problema. 
		\item 
		Que quede claro el problema
		\item 
		Encontrar una ecuación que modele la situación
		\item 
		Conocer los elementos que contiene el problema
		\item 
		Restringirlo
		\item 
		Definirlo
	\end{enumerate}
	Partiendo de este procedimiento, se puede visualizar el problema con mayor facilidad. Otra cosa que también se aprendió en clase fue como hacer un print con una oracion usando los signos de porcentaje y distintas letras a un lado, dependiendo de cada letra, entonces es el resultado que nos imprimirá.
\end{document}