\documentclass[letterpaper, 12pt, oneside]{article}
\usepackage{amsmath}
\usepackage[utf8]{inputenc}
\usepackage{xcolor}
\usepackage{graphicx}
\graphicspath{{Imagenes/}}

	\title{\huge \color{red} Taller de Herramientas Computacionales}
	\author{Brenda Paola García Rivas}
	\date{14.01.19}
	
\begin{document}
	\maketitle
	\begin{center}
		\includegraphics[scale=0.90]{1.png}
	\end{center}
	\newpage
	\title{\Huge Mi Primera Clase de T.H.C\\}
	
	En la primer clase se inició con una introducción de cómo sería el curso de intensivo, se dijeron una serie de puntos que se seguirán durante estas tres semanas, partiendo de esto se dió inicio a la clase formal; en  esta última se vieron una variedad de puntos como:
	\begin{enumerate}
		\item
		Sistemas operativos, algunos de los sistemas operativos son:
		\begin{enumerate}
			\item 
			Linux, el más mencionado y aclamado por matemáticos, computólogos y en general la Facultad de Ciencias encabeza esta lista de sistemas operativos, este ultimo consiste en un software de codigo abierto, el cuál nos permite interactuar con el ordenador de forma que podemos editarlo. Mientras se introdujo este tema,también se hablo de algunos sistemas operativos que son distribuidos por Linux, algunos de éstos son:
			\begin{enumerate}
				\item 
				Ubuntu
				\item 
				Fedora
				\item 
				Debian
			\end{enumerate}
			\item 
			Windows, me parece que es uno de los sistemas operativos más utilizados, puede funcionar en muchos otros dispositivos electrónicos que hagan uso de microprocesadores. Es diferente de Linux, ya que no es un software libre que pueda editarse.
			\item
			iO's. Es un sistema operativo manejado por la empresa Apple
		\end{enumerate}
	\item 
	El segundo punto que se comentó en clase fue el de los lenguajes de programación, para este se explicó que un lenguaje de programacion sirve para órdenes o instrucciones a una computadora para producir datos y existen muchos como:
	\begin{enumerate}
		\item 
		Python 
		\item 
		Java
		\item 
		C++, entre otros
	\end{enumerate}
	\item 
	El tercer punto importante del que se habló fue los registros de los archivos. Algunos de ellos son:
	\begin{enumerate}
		\item
		.exe, Es bueno saber que este archivo no se puede ejecutar o archivar en cualquier lado 
		\item 
		.txt
		\item
		.py 
	\end{enumerate}
\end{enumerate}
\end{document}






