\documentclass[letterpaper, 12pt, oneside]{article}
\usepackage{amsmath}
\usepackage[utf8]{inputenc}
\usepackage{xcolor}
\usepackage{graphicx}
\graphicspath{{Imagenes/}}

	\title{\huge \color{red} Taller de Herramientas Computacionales}
	\author{Brenda Paola García Rivas}
	\date{14.01.19}
	
\begin{document}
	\maketitle
	\begin{center}
		\includegraphics[scale=0.90]{1.png}
	\end{center}
	\newpage
	\title{\Huge Mi Primera Clase de T.H.C\\}
	
	En la primer clase se inició con una introducción de cómo sería el curso de intensivo, se dijeron una serie de puntos que se seguirán durante estas tres semanas, partiendo de esto se dió inicio a la clase formal; en  esta última se vieron una variedad de puntos como:
	\begin{enumerate}
		\item
		Sistemas operativos, algunos de los sistemas operativos son:
		\begin{enumerate}
			\item 
			Linux se vieron tres:
			\begin{enumerate}
				\item 
				Ubuntu
				\item 
				Fedora
			\end{enumerate}
			\item 
			Windows
			\item
			iO's
		\end{enumerate}
	\item 
	Lenguajes de programación, para este se explicó que un enguaje de programacion sirve para... y existen muchos como:
	\begin{enumerate}
		\item 
		Python 
		\item 
		Java
		\item 
		C++
	\end{enumerate}
	\item 
	El tercer punto importante del que se habló fue los registros de los archivos. Algunos de ellos son:
	\begin{enumerate}
		\item
		.exe 
	\end{enumerate}
\end{enumerate}
\end{document}






