\documentclass{article}
\usepackage{xcolor}
\usepackage[utf8]{inputenc}
\usepackage{amsmath}
\usepackage{graphicx}
\graphicspath{{Imagenes/}}

\title{\huge \color{red}Taller de Herramientas Computacionales}
\author{Brenda Paola García Rivas}
\date{24.01.19}

\begin{document}
	\maketitle
	\begin{center}
		\includegraphics [scale=1.5]{1.png}
	\end{center}

	\newpage
	\title{\huge Jueves 24 de Enero. Clase 14}\\
	
	Hoy fue una clase muy interesante, se inició con el problema de fibonacci, el cuál es un problema de recursividad; ayer se vió como resolverlo, sin embargo, esta vez el profesor nos explicó una forma más fácil de hacer los llamados para cada número. 
	Este método implica guardar los resultados en una lista, se modificaría que el programa observe que esté un número para después guardarlo y posteriormente usarlo, pero ¿cómo se sabe que el décimo, onceavo, o el número que sea está en la lista?, pues fácil, teniendo en cuenta la longitud de la lista.
\end{document}